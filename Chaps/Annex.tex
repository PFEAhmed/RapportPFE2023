%\documentclass{report}
%\input{includePackagesChaps.tex}
%\begin{document}
\chapter{Annex}

\section{Project component}
\subsection{Hardware environment}
\subsubsection{Raspberry Pi 3 Model B+}
\FloatBarrier
\begin{figure}[h]
         \centering
        \includegraphics[width=1\textwidth]{ras3B+}
   
        \caption{Raspberry Pi 3 Model B+ }
        \label{Raspberry Pi 3 Model B+ }
    \end{figure}
\FloatBarrier
\FloatBarrier
\begin{table}[htbp]
\centering
\tcbox[width=\linewidth,
  left=0mm,
  right=0mm,
  top=0mm,
  bottom=0mm,
  boxsep=0mm,
  toptitle=0.5mm,
  bottomtitle=0.5mm, title = Raspberry Pi 3 Model B  characteristics \cite{RA16}]{%
\arrayrulecolor{blue!50!black}\renewcommand{\arraystretch}{1.2}%
\begin{tabular}{|l|l|}
\hline
\textbf{Component} & \textbf{Specification} \\ \hline
Processor & 1.4 GHz quad-core BCM2837B0 ARMv8 64bit CPU\\ \hline
Memory & 1 GB RAM \\ \hline
Connectivity & 
\begin{tabular}[c]{@{}l@{}}-dual-band (2.4 GHz and 5 GHz) IEEE 802.11.b/g/n/ac wireless LAN (WiFi)\\ 
- LAN, Bluetooth 5.0, BLE\\ 
- Gigabit Ethernet\\ 
- 4 $\times$ USB 2.0 ports\end{tabular} \\ \hline
GPIO & Standard 40-pin GPIO header (fully backwards-compatible with previous boards) \\ \hline
Video \& Sound & 
\begin{tabular}[c]{@{}l@{}}-  full size HDMI\\ 
- MIPI DSI display port\\ 
- MIPI CSI camera port\\ 
- 4-pole stereo audio and composite video port\end{tabular} \\ \hline
Multimedia & 
\begin{tabular}[c]{@{}l@{}}-MPEG-4 decode (1080p30)\\ 
- H.264 encode(1080p30)\\
- OpenGL ES 1.1, 2.0 graphics
\end{tabular} \\ \hline
SD Card Support &Micro SD format for loading operating system anddata storage \\ \hline
Input Power & 
\begin{tabular}[c]{@{}l@{}}- 5V/2.5A DC via micro USB connector\\ 
- 5V DC via GPIO header (minimum 3A)\\ 
-Power over Ethernet (PoE)–enabled (requires
separate PoE HAT)\end{tabular} \\ \hline
Environment & Operating temperature 0--50ºC \\ \hline
\end{tabular}}
    \addcontentsline{lot}{table}{Raspberry Pi 3 Model B  characteristics \cite{RA16}}
\end{table}

\FloatBarrier
\subsubsection{Raspberry Pi 4 Model B}
\FloatBarrier
\begin{figure}[h]
         \centering
        \includegraphics[width=1\textwidth]{ras4B}
   
        \caption{Raspberry Pi 4 Model B }
        \label{Raspberry Pi 4 Model B }
    \end{figure}
\FloatBarrier
\FloatBarrier
\begin{table}[htbp]
\centering
\tcbox[width=\linewidth,
  left=0mm,
  right=0mm,
  top=0mm,
  bottom=0mm,
  boxsep=0mm,
  toptitle=0.5mm,
  bottomtitle=0.5mm, title = Raspberry Pi 4 model B characteristics \cite{RA19}]{%
\arrayrulecolor{blue!50!black}\renewcommand{\arraystretch}{1.2}%

\begin{tabular}{|l|l|}
\hline
\textbf{Component} & \textbf{Specification} \\ \hline
Processor & Broadcom BCM2711, quad-core Cortex-A72 (ARM v8) 64-bit SoC @ 1.5GHz \\ \hline
Memory & 1GB, 2GB or 4GB LPDDR4 (depending on model) \\ \hline
Connectivity & 
\begin{tabular}[c]{@{}l@{}}- 2.4 GHz and 5.0 GHz IEEE 802.11b/g/n/ac wireless\\ 
- LAN, Bluetooth 5.0, BLE\\ 
- Gigabit Ethernet\\ 
- 2 $\times$ USB 3.0 ports\\ 
- 2 $\times$ USB 2.0 ports\end{tabular} \\ \hline
GPIO & Standard 40-pin GPIO header (fully backwards-compatible with previous boards) \\ \hline
Video \& Sound & 
\begin{tabular}[c]{@{}l@{}}- 2 $\times$ micro HDMI ports (up to 4Kp60 supported)\\ 
- 2-lane MIPI DSI display port\\ 
- 2-lane MIPI CSI camera port\\ 
- 4-pole stereo audio and composite video port\end{tabular} \\ \hline
Multimedia & 
\begin{tabular}[c]{@{}l@{}}- H.265 (4Kp60 decode)\\ 
- H.264 (1080p60 decode, 1080p30 encode)\\ 
- OpenGL ES, 3.0 graphics\end{tabular} \\ \hline
SD Card Support & Micro SD card slot for loading operating system and data storage \\ \hline
Input Power & 
\begin{tabular}[c]{@{}l@{}}- 5V DC via USB-C connector (minimum 3A)\\ 
- 5V DC via GPIO header (minimum 3A)\\ 
- Power over Ethernet (PoE)–enabled\\ 
  (requires separate PoE HAT)\end{tabular} \\ \hline
Environment & Operating temperature 0--50ºC \\ \hline
\end{tabular}}
    \addcontentsline{lot}{table}{ Raspberry Pi 4 model B characteristics \cite{RA19}}
\end{table}
\FloatBarrier
\subsubsection{ESP32-CAM}
\FloatBarrier
\begin{figure}[h]
         \centering
        \includegraphics[width=0.4\textwidth]{esp32cam-}
   
        \caption{ESP32-CAM}
        \label{ESP32-CAM}
    \end{figure}
\FloatBarrier


\begin{table}[h]

\centering
\tcbox[width=\linewidth,
  left=0mm,
  right=0mm,
  top=0mm,
  bottom=0mm,
  boxsep=0mm,
  toptitle=0.5mm,
  bottomtitle=0.5mm, title = ESP32-CAM characteristics \cite{HT}]{%
\arrayrulecolor{blue!50!black}\renewcommand{\arraystretch}{1.2}%



\begin{tabular}{|l|l|}
\hline
\textbf{Component} & \textbf{Specification} \\ \hline
WiFi+Bluetooth module & ESP-32S \\ \hline
Camera module & OV2640 2MP \\ \hline
SPI Flash & 4MB \\ \hline
RAM & Internal 512KB + External 4MB PSRAM \\ \hline
Onboard TF card slot & Supports up to 4G TF card for data storage \\ \hline
Wi-Fi & 802.11b/g/n/e/i \\ \hline
Operating voltage & 3.3/5 Vdc \\ \hline
Power consumption (Flash off) & 180mA@5V \\ \hline
Power consumption (Flash on and brightness max) & 310mA@5V \\ \hline
Power consumption (Modern-Sleep) & as low as 20mA@5V \\ \hline
Power consumption (Light-Sleep) & as low as 6.7mA@5V \\ \hline
Power consumption (Deep-Sleep) & as low as 6mA@5V \\ \hline
Operating temperature & -20 °C ~ 85 °C \\ \hline
Dimensions & 40.5mm x 27mm x 4.5mm \\ \hline
Flash light & LED built-in on board \\ \hline
\end{tabular}}
    \addcontentsline{lot}{table}{ESP32-CAM characteristics \cite{HT}}

\label{table:esp32-cam-characteristics}
\end{table}

\FloatBarrier
\subsubsection{PC}
\FloatBarrier
\begin{figure}[h]
         \centering
        \includegraphics[width=0.4\textwidth]{lenovo-ideapad-gaming-3}
   
        \caption{lenovo ideapad gaming 3}
        \label{fig:lenovo-ideapad-gaming-3}
    \end{figure}
\FloatBarrier
\begin{table}[h]

\centering
\tcbox[width=\linewidth,
  left=0mm,
  right=0mm,
  top=0mm,
  bottom=0mm,
  boxsep=0mm,
  toptitle=0.5mm,
  bottomtitle=0.5mm, title = Specifications of the PC]{%
\arrayrulecolor{blue!50!black}\renewcommand{\arraystretch}{1.2}%



\begin{tabular}{|l|l|}
\hline
\textbf{Component} & \textbf{Specification} \\ \hline
Processor & AMD Ryzen 7 4800H \\ \hline
Memory & 16 GB DDR4 RAM \\ \hline
Graphics & NVIDIA GeForce GTX 1650Ti (4GB GDDR6) \\ \hline
Storage & 512 GB SSD \\ \hline
Operating System & Windows 10 \\ \hline
\end{tabular}}
    \addcontentsline{lot}{table}{Specifications of the PC}

\label{table:pc-specifications}
\end{table}

\FloatBarrier


\subsection{Software environment }
\subsubsection{Software}
\begin{itemize}
  \item Arduino IDE
\FloatBarrier
\begin{figure}[h]
         \centering
        \includegraphics[width=0.2\textwidth]{ArduinoIDELogo}
   
        \caption{Arduino IDE Logo}
        \label{fig:ArduinoIDELogo}
    \end{figure}
\FloatBarrier
\subitem\textbf{Arduino IDE} is an official Arduino software application used for writing, compiling, and uploading code to Arduino microcontrollers. The IDE environment consists of two basic parts: Editor and Compiler and supports both C and C++ languages.\cite{MA18}

 % \item VNC
%\FloatBarrier
%\begin{figure}[h]
   %      \centering
      %  \includegraphics[width=0.2\textwidth]{VNCLogo}
   
        %\caption{VNCLogo}
        %\label{fig:VNCLogo}
    %\end{figure}
%\FloatBarrier
%\subitem\textbf{VNC } (Virtual Network Computing) is a thin-client system that allows users to remotely control and operate another computer or server over a network. It consists of two components: a server that runs on the remote computer, and a client that runs on the local computer.
%\subitem  The server sends screen updates to the client, so the user can see and interact with the remote computer's desktop as if they were sitting right in front of it
  \item Jupter Notebook
\begin{figure}[h]
         \centering
        \includegraphics[width=0.2\textwidth]{jupyterNotebookLogo}
   
        \caption{Jupyter notebook logo}
        \label{fig:Jupyter notebook logo}
    \end{figure}
\FloatBarrier
\subitem\textbf{Jupyter Notebook} is a free, open-source web application that enables interactive computing and data analysis using various programming languages, including Julia, Python, and R.\cite{BMIPC17}
\subitem It allows users to create virtual lab notebooks to support workflows, code, data, and visualizations detailing the research process, making science more open and accessible.\cite{BMIPC17}
  \item Raspberry Pi Imager
\begin{figure}[h]
       \centering
        \includegraphics[width=0.5\textwidth]{raspberry-pi-imagerLogo}
   
        \caption{Raspberry pi imager logo}
        \label{fig:raspberry-pi-imagerLogo}
    \end{figure}
\FloatBarrier
\subitem\textbf{Raspberry Pi Imager}  is a free and open-source software application developed by the Raspberry Pi Foundation.\cite{RSND}
With Raspberry Pi Imager, users can easily choose to install a variety of os, like Raspbian, Ubuntu, Kali Linux, and more,  onto a microSD card that can be used to boot up your Raspberry Pi.
\subitem  It can be downloaded and installed on Windows, Mac, and Linux operating systems.\cite{RSND}
 \item PuTTY
\FloatBarrier
\begin{figure}[h]
       \centering
        \includegraphics[width=0.2\textwidth]{puttyLogo}
   
        \caption{puttyLogo}
        \label{fig:}
    \end{figure}
\FloatBarrier
\subitem\textbf{PuTTY} is a free and open-source terminal emulator, serial console, and network file transfer application. It was originally developed for Windows but is now available on many other operating systems.\cite{PYND}
\subitem PuTTY also supports many network protocols, such as Telnet, rlogin, SSH and raw TCP. It also includes additional features such as session management, SSH key generation, and support for local printing.\cite{PYND}
 \item VNC
\FloatBarrier
\begin{figure}[h]
       \centering
        \includegraphics[width=0.2\textwidth]{VNCLogo}
   
        \caption{VNCLogo}
        \label{fig:VNCLogo}
    \end{figure}
\FloatBarrier
\subitem\textbf{VNC } (Virtual Network Computing) is a thin-client system that allows users to remotely control and operate another computer or server over a network. It consists of two components: a server that runs on the remote computer, and a client that runs on the local computer.\cite{RVND}
\subitem  The server sends screen updates to the client, so the user can see and interact with the remote computer's desktop as if they were sitting right in front of it.\cite{RVND}
  \item Geany
\begin{figure}[h]
         \centering
        \includegraphics[width=0.2\textwidth]{Geanylogo}
   
        \caption{Geany logo}
        \label{fig:Geanylogo}
    \end{figure}
\FloatBarrier
\subitem\textbf{Geany} is a simple and lightweight text editor designed for programmers and developers. It provides features such as syntax highlighting for a wide range of programming languages, code folding, auto-indentation, and built-in support for various programming tools and compilers.\cite{GND}
  \item Google Colab
\begin{figure}[h]
         \centering
        \includegraphics[width=0.4\textwidth]{ColabLogo}
   
        \caption{Google colab logo}
        \label{fig: Google Colab logo}
    \end{figure}
\FloatBarrier
\subitem\textbf{Google Colab} (short for Collaboratory) is a cloud-based platform provided by Google that allows users to run and share Jupyter notebook files for data analysis, machine learning and deep learning tasks. .\cite{GRND}
\subitem It provides access to computing resources such as CPU, GPU,RAM, disk and TPU for free, allowing users to execute complex computational tasks without the need to use their local hardware..\cite{GRND}
\end{itemize}

\subsubsection{programming languages}
\begin{itemize}

\item C++
\FloatBarrier
\begin{figure}[h]
         \centering
        \includegraphics[width=0.2\textwidth]{c++Logo}
   
        \caption{C++ logo}
        \label{fig:c++Logo}
    \end{figure}
\FloatBarrier
\subitem\textbf{C++} is a programming language that is widely used in software development. It is a standardized, general-purpose, and object-oriented language, which means it can be used to create a variety of applications, including system software, device drivers, video games, and desktop applications.\cite{SD20}

\item Python
\begin{figure}[h]
\FloatBarrier
         \centering
        \includegraphics[width=0.2\textwidth]{PythonLogo}
   
        \caption{Python logo}
        \label{fig:PythonLogo}
\FloatBarrier
    \end{figure}
\subitem\textbf{Python} is a popular high-level programming languagethat supports object-oriented, functional, 
and imperative programming styles. scripting language, but can be compiled into computer-readable binary.\cite{SD20}


\end{itemize}
%\begin{thebibliography}{20}
\bibitem{Aw15}M. A. Awad, 2015. \emph{A Comparison between Agile and Traditional Software Development Methodologies 
}.

\bibitem{PGC10}Pete Deemer, Gabrielle Benefield, Craig Larman, Bas Vodde, 2010. \emph{THE
SCRUM PRIMER}.
\bibitem{HMB03}Hans-Erik Eriksson, Magnus Penker, Brian Lyons
, 2003. \emph{UML 2 toolkit}.
\bibitem{LD13}Lenny Delligatti , 2013. \emph{SysML Distilled A Brief Guide}.
\bibitem{MA18} Mohamed FEZARI and Ali Al Dahoud, 2018. \emph{Integrated Development Environment “IDE” For Arduino}.
\bibitem{BMIPC17}Bernadette M. Randles, Milena S. Golshan, Irene V. Pasquetto and Christine L. Borgman, 2017. \emph{Using the Jupyter Notebook as a Tool for Open Science: An Empirical Study}.
\bibitem{SD20}Slobodan Dmitrović, 2020. \emph{ Modern C++ for Absolute Beginners: A Friendly Introduction to C++ Programming Language and C++11 to C++20 Standards}.
\bibitem{CD21}Chakraborty D.,2021. \emph{ OpenCV Contour Approximation ( cv2.approxPolyDP ),}.
\bibitem{WG10}Willow Garage, 2010. \emph{OpenCV Reference Manual v2.2}.
\bibitem{HT}Handson Technology. \emph{ ESP32-CAM WiFi+Bluetooth+Camera Module Datasheet ),}.
\bibitem{E19}Espressif,2019. \emph{ ESP32-WROOM-32 Datasheet),}.
\bibitem{RSND}Raspberry Pi Software,(n.d.). \emph{https://www.raspberrypi.com/software/}
\bibitem{PYND}PuTTY. (n.d.). \emph{ https://www.putty.org/}
\bibitem{RVND}RealVNC. (n.d.). \emph{ https://discover.realvnc.com/what-is-vnc-remote-access-technology}
\bibitem{GND}Geany. (n.d.). \emph{https://www.geany.org/}
\bibitem{GRND}Google Research . (n.d.). \emph{https://research.google.com/colaboratory/faq.html}
\bibitem{YC21}Pappu Kumar YadavJ. Alex ThomassonStephen W. SearcyRobert G. HardinUlisses Braga-NetoSorin C. PopescuDaniel E. MartinRoberto RodriguezKarem MezaJuan EncisoJorge Solórzano DiazTianyi Wang, 2021. \emph{ Assessing The Performance of YOLOv5 Algorithm For Detecting Volunteer Cotton Plants in Corn Fields at Three Different Growth Stages}.
\bibitem{KGMM21}Kathrin Blagec  , Georg Dorffner  , Milad Moradi , Matthias Samwald , 2020. \emph{ A critical analysis of metrics used for measuring progress in artificial intelligence }.
\bibitem{MG21}Marko Horvat, Gordan Gledec, 2022. \emph{A comparative study of YOLOv5 models performance for image localization and classification  }.
\bibitem{MG21}Marko Horvat, Gordan Gledec, 2022. \emph{A comparative study of YOLOv5 models performance for image localization and classification  }.
\bibitem{U23}ultralytics, 2023. \emph{yolov5 github repository : https://github.com/ultralytics/yolov5 }.
\bibitem{HW22}hitechwhizz, 2022. \emph{https://www.hitechwhizz.com/2021/09/advantages-and-disadvantages-drawbacks-benefits-of-static-ip-address.html.html.html}.
%https://pyimagesearch.com/2021/10/06/opencv-contour-approximation/



\end{thebibliography}





%\end{document}