%\documentclass{report}
%\usepackage[left=1.25in,right=1.25in,top=1in,bottom=1.5in]{geometry} % set different margins
\usepackage{placeins}
\usepackage{graphicx}
\usepackage{hyperref}
\usepackage{bookmark}
\usepackage{titlesec}
\usepackage[T1]{fontenc}
%\usepackage{amsmath}
\usepackage{tabularx}
\usepackage{ifpdf}
\usepackage{ae}
\usepackage{tikz}
\usepackage{colortbl}
\usepackage{caption}

\usepackage{wrapfig}
%\usepackage{../Packages/tikz-uml}% this package is for uml diagrams
\usepackage{Packages/tikz-uml}% this package is for uml diagrams
\usepackage{caption} % load caption package
\usepackage{enumitem}
\usepackage{listings}
\usepackage{amsmath}
\usepackage{subcaption}
\usepackage{xcolor}
\usepackage{listings}
\usepackage{tcolorbox}

\tcbuselibrary{listings}
\definecolor{LightGray}{gray}{0.95}
\usepackage{tcolorbox}
\definecolor{dkgreen}{RGB}{0,128,0}
\NewTotalTCBox{\commandbox}{ s v }
{verbatim,colupper=white,colback=black!75!white,colframe=black}
{\IfBooleanT{#1}{\textcolor{red}{\ttfamily\bfseries > }}%
\lstinline[language=command.com,keywordstyle=\color{blue!35!white}\bfseries]^#2^}
\lstset{ 
  backgroundcolor=\color{LightGray},
  basicstyle=\ttfamily,
  breaklines=true,
  captionpos=b,
  language=bash,
  frame=single
}


\tcbset{colframe=blue!50!black,colback=white,colupper=black,
fonttitle=\bfseries,nobeforeafter,center title}
\definecolor{mauve}{RGB}{224,176,255}

\usetikzlibrary{positioning}
\lstset{frame=tb,
  language=Python,
  aboveskip=3mm,
  belowskip=3mm,
  showstringspaces=false,
  columns=flexible,
  basicstyle={\small\ttfamily},
  numbers=none,
  numberstyle=\tiny\color{gray},
  keywordstyle=\color{blue},
  commentstyle=\color{dkgreen},
  stringstyle=\color{mauve},
  breaklines=true,
  breakatwhitespace=true,
  tabsize=3
}
\setcounter{secnumdepth}{4}
\setcounter{tocdepth}{4}
\addcontentsline{toc}{section}{Introduction}

\renewcommand{\thesection}{\Roman{section}} 
\renewcommand{\thesubsection}{\arabic{subsection}}
\titlespacing{\subsection}{5pt}{1em}{0pt}
\graphicspath{ {./Ilustrations}{./}{./images}}
%\graphicspath{ {../Ilustrations}{../images}}

\tikzstyle{usecase}=[ellipse, fill=white, draw=black, thick, inner sep=2pt, text centered]
%\begin{document}



\chapter{Sprint 2: Stabilizing the System}
\section{Introduction}
In this sprint, we made several significant changes to our project that have improved its functionality and usability. First, we decided to switch from using a traditional router to using a Raspberry Pi as an access point, which allowed us to have more control over our network and make our setup more portable. We also made the decision to give our ESP32-CAM server a static IP address, which helped to improve its stability and reliability.

In addition, we identified an issue with the position of the logo and the font. To address this, we used image processing with the addition of a new class to the YOLOV5s to help us accurately identify and classify these types of objects. This change has significantly improved the accuracy and reliability of our system, and we are confident that it will provide a better user experience for our end-users.

Furthermore, I am excited to announce that I automated the main script for inference and I fixed the raspberry pi's temperature issue.

Overall, we are pleased with the progress we have made in this sprint and believe that these changes will have a positive impact on our project moving forward.
\section{Workflow}
\subsection{Network improvements}
\subsubsection{Turning the raspberry pi into an access}
There are several reasons I turned the Raspberry Pi into an access point rather than relying on a normal router. First, Flexibility: By setting up a Raspberry Pi as an access point, you have more control over your network and can customize it to meet your specific needs. Second, Cost: Using a Raspberry Pi as an access point can be cheaper than buying a dedicated router, especially if you already have a spare Raspberry Pi lying around.Third,Portability: A Raspberry Pi is small and portable, making it easy to set up an access point on the go. This can be useful if you need to quickly set up a network in a remote location, such as a camping trip or a field research project.Finally, Integration: If I'am already using a Raspberry Pi for other purposes, so I can easily integrate the access point functionality into your existing setup.
\subsubsection{Giving static address to ESP32-CAM}
There are several advantages of assigning a static IP address to an ESP32-CAM device compared to using a dynamic IP address obtained from a DHCP server such as Stability: With a static IP address, the ESP32-CAM device will always have the same IP address, which ensures stability in the network connection. In contrast, a dynamic IP address may change periodically, which can cause connectivity issues if the device's IP address changes without the network administrator's knowledge
and in addition to that Easier network access: A static IP address makes it easier to access the ESP32-CAM device remotely. Since the IP address remains constant, there is no need to check the DHCP server for the device's current IP address, which can be especially beneficial in situations where you need to access the device frequently.
\subsection{Software improvement}
\subsubsection{Automating the main script for inference}
Once we installed our product in the factory, we noticed that the main script inside it required human intervention to work properly. This was problematic because it meant that an operator had to be present to initiate the script every time the system was started up. We knew that this was not an efficient or sustainable solution, so we began exploring ways to automate the script.

After some research and testing, we were able to configure the script to run automatically at the start of the Raspberry Pi. This allowed the system to operate without any human intervention, reducing the need for manual input and increasing the overall efficiency of the system. With the script now automated, the system could start up and run smoothly without any manual intervention, freeing up our operators to focus on other tasks.

The benefits of automating a script are numerous. Automating a script can reduce the need for manual input, saving time and increasing efficiency. In addition, automating a script can reduce the potential for human error, ensuring that the system operates reliably and accurately. Furthermore, automating a script can help to reduce costs associated with hiring additional staff or training existing staff on manual processes.

Overall, the automation of our script was a significant improvement to our system, and we are confident that it will continue to improve the efficiency and reliability of our product moving forward. By automating this process, we have created a more sustainable and efficient system that will benefit both our team and our end-users.
\subsubsection{Detection improvements}
\subsection{Fixing the temperature issue}
After installing the first version of our product in the factory, we noticed that the temperature of the Raspberry Pi was rising significantly. This was likely due to the hot environment and the fact that the Raspberry Pi was running YOLOv5, which requires a significant amount of processing power. We knew that this issue could impact the reliability and longevity of our product, so we immediately began exploring solutions.

After considering several options, we decided to install a fan to cool the Raspberry Pi. This solution proved to be effective in reducing the temperature of the Raspberry Pi and maintaining stable performance. By keeping the Raspberry Pi cool, we were able to prevent it from overheating and potentially causing damage to the system. Additionally, we found that cooling the Raspberry Pi improved the overall performance of our system, allowing it to run more smoothly and efficiently.

In general, the benefits of cooling a Raspberry Pi are numerous. Cooling a Raspberry Pi can improve its longevity, reduce the risk of damage, and ensure that it runs smoothly and efficiently. Overheating can cause a Raspberry Pi to crash, freeze, or even become permanently damaged. By keeping the Raspberry Pi cool, we can prevent these issues and ensure that our product is reliable and durable. Furthermore, cooling a Raspberry Pi can help to extend its lifespan, which can ultimately save time and money in the long run.


%\begin{thebibliography}{20}
\bibitem{Aw15}M. A. Awad, 2015. \emph{A Comparison between Agile and Traditional Software Development Methodologies 
}.

\bibitem{PGC10}Pete Deemer, Gabrielle Benefield, Craig Larman, Bas Vodde, 2010. \emph{THE
SCRUM PRIMER}.
\bibitem{HMB03}Hans-Erik Eriksson, Magnus Penker, Brian Lyons
, 2003. \emph{UML 2 toolkit}.
\bibitem{LD13}Lenny Delligatti , 2013. \emph{SysML Distilled A Brief Guide}.
\bibitem{MA18} Mohamed FEZARI and Ali Al Dahoud, 2018. \emph{Integrated Development Environment “IDE” For Arduino}.
\bibitem{BMIPC17}Bernadette M. Randles, Milena S. Golshan, Irene V. Pasquetto and Christine L. Borgman, 2017. \emph{Using the Jupyter Notebook as a Tool for Open Science: An Empirical Study}.
\bibitem{SD20}Slobodan Dmitrović, 2020. \emph{ Modern C++ for Absolute Beginners: A Friendly Introduction to C++ Programming Language and C++11 to C++20 Standards}.
\bibitem{CD21}Chakraborty D.,2021. \emph{ OpenCV Contour Approximation ( cv2.approxPolyDP ),}.
\bibitem{WG10}Willow Garage, 2010. \emph{OpenCV Reference Manual v2.2}.
\bibitem{HT}Handson Technology. \emph{ ESP32-CAM WiFi+Bluetooth+Camera Module Datasheet ),}.
\bibitem{E19}Espressif,2019. \emph{ ESP32-WROOM-32 Datasheet),}.
\bibitem{RSND}Raspberry Pi Software,(n.d.). \emph{https://www.raspberrypi.com/software/}
\bibitem{PYND}PuTTY. (n.d.). \emph{ https://www.putty.org/}
\bibitem{RVND}RealVNC. (n.d.). \emph{ https://discover.realvnc.com/what-is-vnc-remote-access-technology}
\bibitem{GND}Geany. (n.d.). \emph{https://www.geany.org/}
\bibitem{GRND}Google Research . (n.d.). \emph{https://research.google.com/colaboratory/faq.html}
\bibitem{YC21}Pappu Kumar YadavJ. Alex ThomassonStephen W. SearcyRobert G. HardinUlisses Braga-NetoSorin C. PopescuDaniel E. MartinRoberto RodriguezKarem MezaJuan EncisoJorge Solórzano DiazTianyi Wang, 2021. \emph{ Assessing The Performance of YOLOv5 Algorithm For Detecting Volunteer Cotton Plants in Corn Fields at Three Different Growth Stages}.
\bibitem{KGMM21}Kathrin Blagec  , Georg Dorffner  , Milad Moradi , Matthias Samwald , 2020. \emph{ A critical analysis of metrics used for measuring progress in artificial intelligence }.
\bibitem{MG21}Marko Horvat, Gordan Gledec, 2022. \emph{A comparative study of YOLOv5 models performance for image localization and classification  }.
\bibitem{MG21}Marko Horvat, Gordan Gledec, 2022. \emph{A comparative study of YOLOv5 models performance for image localization and classification  }.
\bibitem{U23}ultralytics, 2023. \emph{yolov5 github repository : https://github.com/ultralytics/yolov5 }.
\bibitem{HW22}hitechwhizz, 2022. \emph{https://www.hitechwhizz.com/2021/09/advantages-and-disadvantages-drawbacks-benefits-of-static-ip-address.html.html.html}.
%https://pyimagesearch.com/2021/10/06/opencv-contour-approximation/



\end{thebibliography}





%\end{document}