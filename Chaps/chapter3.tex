%\documentclass{report}
%\usepackage[left=1.25in,right=1.25in,top=1in,bottom=1.5in]{geometry} % set different margins
\usepackage{placeins}
\usepackage{graphicx}
\usepackage{hyperref}
\usepackage{bookmark}
\usepackage{titlesec}
\usepackage[T1]{fontenc}
%\usepackage{amsmath}
\usepackage{tabularx}
\usepackage{ifpdf}
\usepackage{ae}
\usepackage{tikz}
\usepackage{colortbl}
\usepackage{caption}

\usepackage{wrapfig}
%\usepackage{../Packages/tikz-uml}% this package is for uml diagrams
\usepackage{Packages/tikz-uml}% this package is for uml diagrams
\usepackage{caption} % load caption package
\usepackage{enumitem}
\usepackage{listings}
\usepackage{amsmath}
\usepackage{subcaption}
\usepackage{xcolor}
\usepackage{listings}
\usepackage{tcolorbox}

\tcbuselibrary{listings}
\definecolor{LightGray}{gray}{0.95}
\usepackage{tcolorbox}
\definecolor{dkgreen}{RGB}{0,128,0}
\NewTotalTCBox{\commandbox}{ s v }
{verbatim,colupper=white,colback=black!75!white,colframe=black}
{\IfBooleanT{#1}{\textcolor{red}{\ttfamily\bfseries > }}%
\lstinline[language=command.com,keywordstyle=\color{blue!35!white}\bfseries]^#2^}
\lstset{ 
  backgroundcolor=\color{LightGray},
  basicstyle=\ttfamily,
  breaklines=true,
  captionpos=b,
  language=bash,
  frame=single
}


\tcbset{colframe=blue!50!black,colback=white,colupper=black,
fonttitle=\bfseries,nobeforeafter,center title}
\definecolor{mauve}{RGB}{224,176,255}

\usetikzlibrary{positioning}
\lstset{frame=tb,
  language=Python,
  aboveskip=3mm,
  belowskip=3mm,
  showstringspaces=false,
  columns=flexible,
  basicstyle={\small\ttfamily},
  numbers=none,
  numberstyle=\tiny\color{gray},
  keywordstyle=\color{blue},
  commentstyle=\color{dkgreen},
  stringstyle=\color{mauve},
  breaklines=true,
  breakatwhitespace=true,
  tabsize=3
}
\setcounter{secnumdepth}{4}
\setcounter{tocdepth}{4}
\addcontentsline{toc}{section}{Introduction}

\renewcommand{\thesection}{\Roman{section}} 
\renewcommand{\thesubsection}{\arabic{subsection}}
\titlespacing{\subsection}{5pt}{1em}{0pt}
\graphicspath{ {./Ilustrations}{./}{./images}}
%\graphicspath{ {../Ilustrations}{../images}}

\tikzstyle{usecase}=[ellipse, fill=white, draw=black, thick, inner sep=2pt, text centered]
%\begin{document}



\chapter{Sprint Two: Raspberry pi 4 and ESP32 Cam Integration for Logo Detection}
\section*{Introduction}
\section{Specification of requirements}
In this section, we introduce the different actors as well as the functional and non-functional requirements.
\subsection{actors identification }
\subsection{Description of functional requirements}
\begin{itemize}
\item The ESP32-CAM board should be able to capture a live stream of pictures or videos and make them available through its built-in server.
\item The raspberry pi should be able to connect to the server on the ESP32-CAM board and retrieve the images or videos.
\item The raspberry pi should be able to process the images to determine if a specific logo is present or not.
\item The raspberry pi should be able to process the images to determine if a specific logo is flipped on the x axis.
\item The raspberry pi should be able to process the images to determine if a specific logo is in the first or the second half of the product.
\item The raspberry pi should be able to provide some form of feedback or output indicating whether the logo is present and if it is flipped or not.
\end{itemize}
\subsection{Description of non-functional requirements}
The requirements do not stop at the functional level but tend towards requirements that contribute to better quality of the application. The most important ones are:
\begin{itemize}
\item \textbf{Reliability:} The system should be able to consistently capture and process images accurately.
\item \textbf{Performance:} The system should be able to process images quickly and without noticeable lag or delay.
\item \textbf{Security:} The system should have appropriate security measures in place to prevent unauthorized access to the servers and data.
\item \textbf{Scalability:} The system should be able to handle multiple simultaneous connections and requests from clients without compromising its performance.
\item \textbf{Maintainability:} The system should be easy to maintain and update, with clear and well-documented code and configuration.
\item \textbf{Compatibility:} The system should be compatible with a wide range of devices and platforms.
\item \textbf{Usability:} The system should be easy to use and understand for both technical and non-technical users.
\end{itemize}

\section{Modeling languages diagrams}
\subsection{SysMl}
\FloatBarrier
\begin{figure}[htbp]
    \centering
    \begin{tikzpicture}
        \node[draw, fill=white, minimum height=2cm, minimum width=2cm, line width=2pt] (ESP32-CAM) at (-4,0) {};
        \node[draw, fill=white, minimum height=2cm, minimum width=2cm, line width=2pt] (Raspberry_pi) at (-4,-5) {};
      %  \umlactor[x=-4, y=0, scale=2,draw=black, line width=2pt]{ESP32-CAM}
        %\umlactor[x=-4, y=-5, scale=2,draw=black, line width=2pt]{ESP32}
  \umlusecase[x=-2, y=4, width=3cm, name=server,fill=white,draw=black, line width=2pt]{\begin{tabular}{c}Create a server\\         \end{tabular}}
  \umlusecase[x=7, y=4, width=3cm, name=connect,fill=white,draw=black, line width=2pt]{\begin{tabular}{c}Connect to a wifi\\         \end{tabular}}
  \umlusecase[x=5, y=-1, width=3cm, name=connect2,fill=white,draw=black, line width=2pt]{\begin{tabular}{c}Connect to a wifi\\         \end{tabular}}
  \umlusecase[x=0, y=-2, width=3cm, name=connect_to_server,fill=white,draw=black, line width=2pt]{\begin{tabular}{c}Connect to the server \\         \end{tabular}}
        \umlusecase[x=0, y=2, width=3cm, name=footages,fill=white,draw=black, line width=2pt]{\begin{tabular}{c}Take live footages\\         \end{tabular}}
        \umlusecase[x=0, y=-4, width=3cm, name=predict,fill=white,draw=black, line width=2pt]{\begin{tabular}{c}Predict the state\\of the logo\\\end{tabular}}
        \umlusecase[x=5, y=3, width=3cm, name=extend1,fill=white,draw=black, line width=2pt]{\begin{tabular}{c}Take photos\end{tabular}} 
        \umlusecase[x=5, y=1, width=3cm, name=extend2,fill=white,draw=black, line width=2pt]{\begin{tabular}{c}Take vidoes\end{tabular}}  
        \umlusecase[x=4, y=-11, width=3cm, name=Present logo,fill=white,draw=black, line width=2pt]{\begin{tabular}{c}Present logo\end{tabular}} 
        \umlusecase[x=5, y=-4, width=3cm, name=Flipped on the x axis,fill=white,draw=black, line width=2pt]{\begin{tabular}{c}Logo flipped on \\ the x  axis\end{tabular}}  
   \umlusecase[x=5.5, y=-6, width=4.25cm, name=first,fill=white,draw=black, line width=2pt]{\begin{tabular}{c}Logo flipped on \\ the first half of the product\end{tabular}}  
  \umlusecase[x=5.5, y=-8, width=4.25cm, name=second,fill=white,draw=black, line width=2pt]{\begin{tabular}{c}Logo flipped on \\ the second half of the product\end{tabular}}  
        \umlusecase[x=5, y=-2.5, width=3cm, name=No logo,fill=white,draw=black, line width=2pt]{\begin{tabular}{c}No logo\end{tabular}}  
%\draw[->] (connect_to_server.west) -| node[pos=1, above] {<<secondary>>} ([xshift=-4mm]ESP32-CAM.east) |- (connect_to_server.east);


  \umlinclude{server}{connect}
 \umlinclude{connect_to_server}{predict}
 \umlinclude{connect2}{connect_to_server}

        \umlinherit{Present logo}{predict}
        \umlinherit{Flipped on the x axis}{predict}
        \umlinherit{No logo}{predict} 
        \umlinherit{first}{predict}
        \umlinherit{second}{predict}  
        \umlinherit{extend1}{footages}  
        \umlinherit{extend2}{footages} 

        \node[fit=(predict), inner ysep=4ex] {}; % increase height of predict use case
        \node[fit=(Flipped on the x axis), inner ysep=4ex]{};
        \node[fit=(first), inner xsep=8ex]{};
        \umlassoc{ESP32-CAM}{footages}
  \umlassoc{ESP32-CAM}{server}
\node[below] at (ESP32-CAM.south) {ESP32-CAM};
        \node[below] at (Raspberry_pi.south) {Raspberry pi};
        \umlassoc{Raspberry_pi}{predict}
    \end{tikzpicture}
    \caption{Use case Diagram for sprint version 1}
    \label{fig:usecase-sprint0}
\end{figure}



\FloatBarrier




\section{Project component}
\subsection{Hardware environment}
\subsubsection{ESP32-CAM}
\FloatBarrier
\begin{figure}[h]
         \centering
        \includegraphics[width=0.4\textwidth]{esp32cam}
   
        \caption{ESP32-CAM}
        \label{fig:esp32cam}
    \end{figure}
\FloatBarrier
\FloatBarrier
\begin{table}[h]
\centering
\begin{tabular}{|l|l|}
\hline
\textbf{Component} & \textbf{Specification} \\ \hline
WiFi+Bluetooth module & ESP-32S \\ \hline
Camera module & OV2640 2MP \\ \hline
SPI Flash & 4MB \\ \hline
RAM & Internal 512KB + External 4MB PSRAM \\ \hline
Onboard TF card slot & Supports up to 4G TF card for data storage \\ \hline
Wi-Fi & 802.11b/g/n/e/i \\ \hline
Operating voltage & 3.3/5 Vdc \\ \hline
Power consumption (Flash off) & 180mA@5V \\ \hline
Power consumption (Flash on and brightness max) & 310mA@5V \\ \hline
Power consumption (Modern-Sleep) & as low as 20mA@5V \\ \hline
Power consumption (Light-Sleep) & as low as 6.7mA@5V \\ \hline
Power consumption (Deep-Sleep) & as low as 6mA@5V \\ \hline
Operating temperature & -20 °C ~ 85 °C \\ \hline
Dimensions & 40.5mm x 27mm x 4.5mm \\ \hline
Flash light & LED built-in on board \\ \hline
\end{tabular}
\caption{ESP32-CAM characteristics \cite{HT}}
\label{table:esp32-cam-characteristics}
\end{table}
\FloatBarrier

\subsubsection{PC}
\FloatBarrier
\begin{figure}[h]
         \centering
        \includegraphics[width=0.4\textwidth]{lenovo-ideapad-gaming-3}
   
        \caption{lenovo ideapad gaming 3}
        \label{fig:lenovo-ideapad-gaming-3}
    \end{figure}
\FloatBarrier

\begin{table}[h]
\centering
\begin{tabular}{|l|l|}
\hline
\textbf{Component} & \textbf{Specification} \\ \hline
Processor & AMD Ryzen 7 4800H \\ \hline
Memory & 16 GB DDR4 RAM \\ \hline
Graphics & NVIDIA GeForce GTX 1650Ti (4GB GDDR6) \\ \hline
Storage & 512 GB SSD \\ \hline
Operating System & Windows 10 \\ \hline
\end{tabular}
\caption{Specifications of the PC}
\label{table:pc-specifications}
\end{table}

\subsection{Software environment }
\subsubsection{Software}
\begin{itemize}
  %\item Arduino IDE
%\FloatBarrier
%\begin{figure}[h]
   %      \centering
      %  \includegraphics[width=0.2\textwidth]{ArduinoIDELogo}
   
        %\caption{Arduino IDE Logo}
        %\label{fig:ArduinoIDELogo}
    %\end{figure}
%\FloatBarrier
%\subitem\textbf{Arduino IDE} is an official Arduino software application used for writing, compiling, and uploading code to Arduino microcontrollers. The IDE environment consists of two basic parts: Editor and Compiler and supports both C and C++ languages.\cite{MA18}

  \item Raspberry Pi Imager
\begin{figure}[h]
       \centering
        \includegraphics[width=0.5\textwidth]{raspberry-pi-imagerLogo}
   
        \caption{Raspberry pi imager logo}
        \label{fig:raspberry-pi-imagerLogo}
    \end{figure}
\FloatBarrier
\subitem\textbf{Raspberry Pi Imager}  is a free and open-source software application developed by the Raspberry Pi Foundation.\cite{RSND}
With Raspberry Pi Imager, users can easily choose to install a variety of os, like Raspbian, Ubuntu, Kali Linux, and more,  onto a microSD card that can be used to boot up your Raspberry Pi.
\subitem  It can be downloaded and installed on Windows, Mac, and Linux operating systems.\cite{RSND}
 \item PuTTY
\FloatBarrier
\begin{figure}[h]
       \centering
        \includegraphics[width=0.2\textwidth]{puttyLogo}
   
        \caption{puttyLogo}
        \label{fig:}
    \end{figure}
\FloatBarrier
\subitem\textbf{PuTTY} is a free and open-source terminal emulator, serial console, and network file transfer application. It was originally developed for Windows but is now available on many other operating systems.\cite{PYND}
\subitem PuTTY also supports many network protocols, such as Telnet, rlogin, SSH and raw TCP. It also includes additional features such as session management, SSH key generation, and support for local printing.\cite{PYND}
 \item VNC
\FloatBarrier
\begin{figure}[h]
       \centering
        \includegraphics[width=0.2\textwidth]{VNCLogo}
   
        \caption{VNCLogo}
        \label{fig:VNCLogo}
    \end{figure}
\FloatBarrier
\subitem\textbf{VNC } (Virtual Network Computing) is a thin-client system that allows users to remotely control and operate another computer or server over a network. It consists of two components: a server that runs on the remote computer, and a client that runs on the local computer.\cite{RVND}
\subitem  The server sends screen updates to the client, so the user can see and interact with the remote computer's desktop as if they were sitting right in front of it.\cite{RVND}
  \item Geany
\begin{figure}[h]
         \centering
        \includegraphics[width=0.2\textwidth]{Geanylogo}
   
        \caption{Geany logo}
        \label{fig:Geanylogo}
    \end{figure}
\FloatBarrier
\subitem\textbf{Geany} is a simple and lightweight text editor designed for programmers and developers. It provides features such as syntax highlighting for a wide range of programming languages, code folding, auto-indentation, and built-in support for various programming tools and compilers.\cite{GND}
  \item Google Colab
\begin{figure}[h]
         \centering
        \includegraphics[width=0.4\textwidth]{ColabLogo}
   
        \caption{Google colab logo}
        \label{fig: Google Colab logo}
    \end{figure}
\FloatBarrier
\subitem\textbf{Google Colab} (short for Collaboratory) is a cloud-based platform provided by Google that allows users to run and share Jupyter notebook files for data analysis, machine learning and deep learning tasks. .\cite{GRND}
\subitem It provides access to computing resources such as CPU, GPU,RAM, disk and TPU for free, allowing users to execute complex computational tasks without the need to use their local hardware..\cite{GRND}
\end{itemize}
\subsubsection{programming languages}
\begin{itemize}

\item C++
\FloatBarrier
\begin{figure}[h]
         \centering
        \includegraphics[width=0.2\textwidth]{c++Logo}
   
        \caption{C++ logo}
        \label{fig:c++Logo}
    \end{figure}
\FloatBarrier
\subitem\textbf{C++} is a programming language that is widely used in software development. It is a standardized, general-purpose, and object-oriented language, which means it can be used to create a variety of applications, including system software, device drivers, video games, and desktop applications.\cite{SD20}

\item Python
\FloatBarrier
\begin{figure}[h]
         \centering
        \includegraphics[width=0.2\textwidth]{PythonLogo}
   
        \caption{Python logo}
        \label{fig:PythonLogo}
    \end{figure}
\FloatBarrier
\subitem\textbf{Python} is a popular high-level programming languagethat supports object-oriented, functional, 
and imperative programming styles. scripting language, but can be compiled into computer-readable binary.\cite{SD20}


\end{itemize}

\subsection{Workflow}

%\begin{thebibliography}{20}
\bibitem{Aw15}M. A. Awad, 2015. \emph{A Comparison between Agile and Traditional Software Development Methodologies 
}.

\bibitem{PGC10}Pete Deemer, Gabrielle Benefield, Craig Larman, Bas Vodde, 2010. \emph{THE
SCRUM PRIMER}.
\bibitem{HMB03}Hans-Erik Eriksson, Magnus Penker, Brian Lyons
, 2003. \emph{UML 2 toolkit}.
\bibitem{LD13}Lenny Delligatti , 2013. \emph{SysML Distilled A Brief Guide}.
\bibitem{MA18} Mohamed FEZARI and Ali Al Dahoud, 2018. \emph{Integrated Development Environment “IDE” For Arduino}.
\bibitem{BMIPC17}Bernadette M. Randles, Milena S. Golshan, Irene V. Pasquetto and Christine L. Borgman, 2017. \emph{Using the Jupyter Notebook as a Tool for Open Science: An Empirical Study}.
\bibitem{SD20}Slobodan Dmitrović, 2020. \emph{ Modern C++ for Absolute Beginners: A Friendly Introduction to C++ Programming Language and C++11 to C++20 Standards}.
\bibitem{CD21}Chakraborty D.,2021. \emph{ OpenCV Contour Approximation ( cv2.approxPolyDP ),}.
\bibitem{WG10}Willow Garage, 2010. \emph{OpenCV Reference Manual v2.2}.
\bibitem{HT}Handson Technology. \emph{ ESP32-CAM WiFi+Bluetooth+Camera Module Datasheet ),}.
\bibitem{E19}Espressif,2019. \emph{ ESP32-WROOM-32 Datasheet),}.
\bibitem{RSND}Raspberry Pi Software,(n.d.). \emph{https://www.raspberrypi.com/software/}
\bibitem{PYND}PuTTY. (n.d.). \emph{ https://www.putty.org/}
\bibitem{RVND}RealVNC. (n.d.). \emph{ https://discover.realvnc.com/what-is-vnc-remote-access-technology}
\bibitem{GND}Geany. (n.d.). \emph{https://www.geany.org/}
\bibitem{GRND}Google Research . (n.d.). \emph{https://research.google.com/colaboratory/faq.html}
\bibitem{YC21}Pappu Kumar YadavJ. Alex ThomassonStephen W. SearcyRobert G. HardinUlisses Braga-NetoSorin C. PopescuDaniel E. MartinRoberto RodriguezKarem MezaJuan EncisoJorge Solórzano DiazTianyi Wang, 2021. \emph{ Assessing The Performance of YOLOv5 Algorithm For Detecting Volunteer Cotton Plants in Corn Fields at Three Different Growth Stages}.
\bibitem{KGMM21}Kathrin Blagec  , Georg Dorffner  , Milad Moradi , Matthias Samwald , 2020. \emph{ A critical analysis of metrics used for measuring progress in artificial intelligence }.
\bibitem{MG21}Marko Horvat, Gordan Gledec, 2022. \emph{A comparative study of YOLOv5 models performance for image localization and classification  }.
\bibitem{MG21}Marko Horvat, Gordan Gledec, 2022. \emph{A comparative study of YOLOv5 models performance for image localization and classification  }.
\bibitem{U23}ultralytics, 2023. \emph{yolov5 github repository : https://github.com/ultralytics/yolov5 }.
\bibitem{HW22}hitechwhizz, 2022. \emph{https://www.hitechwhizz.com/2021/09/advantages-and-disadvantages-drawbacks-benefits-of-static-ip-address.html.html.html}.
%https://pyimagesearch.com/2021/10/06/opencv-contour-approximation/



\end{thebibliography}





%\end{document}