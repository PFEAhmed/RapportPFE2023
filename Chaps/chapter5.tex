%\documentclass{report}
%\usepackage[left=1.25in,right=1.25in,top=1in,bottom=1.5in]{geometry} % set different margins
\usepackage{placeins}
\usepackage{graphicx}
\usepackage{hyperref}
\usepackage{bookmark}
\usepackage{titlesec}
\usepackage[T1]{fontenc}
%\usepackage{amsmath}
\usepackage{tabularx}
\usepackage{ifpdf}
\usepackage{ae}
\usepackage{tikz}
\usepackage{Packages/tikz-uml}% this package is for uml diagrams
\usepackage{caption} % load caption package
\usepackage{enumitem}
\usepackage{listings}
\usepackage{amsmath}
\usepackage{subcaption}
\usetikzlibrary{positioning}
\lstset{frame=tb,
  language=Python,
  aboveskip=3mm,
  belowskip=3mm,
  showstringspaces=false,
  columns=flexible,
  basicstyle={\small\ttfamily},
  numbers=none,
  numberstyle=\tiny\color{gray},
  keywordstyle=\color{blue},
  commentstyle=\color{dkgreen},
  stringstyle=\color{mauve},
  breaklines=true,
  breakatwhitespace=true,
  tabsize=3
}
\setcounter{secnumdepth}{3}
\setcounter{tocdepth}{3}
\addcontentsline{toc}{section}{Introduction}

\renewcommand{\thesection}{\Roman{section}} 
\renewcommand{\thesubsection}{\arabic{subsection}}
\titlespacing{\subsection}{5pt}{1em}{0pt}
\graphicspath{ {./Ilustrations}{./}{./images} }

\tikzstyle{usecase}=[ellipse, fill=white, draw=black, thick, inner sep=2pt, text centered]
%\begin{document}

%\chapter{Sprint0 :  <{}< version 0 >{}>}

\chapter{Sprint 4: PCB Board and Case Design}
\section{ PCB board  design}
There are several compelling reasons why using a printed circuit board (PCB) to mount and connect components is preferred over relying solely on a breadboard and wires. Firstly, a PCB offers enhanced reliability and stability compared to a breadboard. The components are securely soldered onto the PCB, ensuring a more secure and permanent connection. This eliminates the risk of loose connections or accidental disconnections that can occur with breadboards and wires, especially in situations where the device may be subjected to movement or vibrations.

Secondly, a PCB provides a more compact and organized layout for the components. With a breadboard, components and wires are often scattered and loosely arranged, making it challenging to maintain a clean and efficient design. PCBs allow for a streamlined and optimized placement of components, reducing the overall size of the circuit and facilitating easier troubleshooting and maintenance.

Furthermore, PCBs offer improved electrical performance and reduced signal interference. The copper traces on a PCB provide precise pathways for electrical signals, minimizing signal loss and crosstalk. This is particularly crucial in high-frequency applications or circuits that require precise signal transmission. In contrast, breadboards and wires introduce additional resistance and capacitance, which can degrade signal quality and introduce noise.

Another significant advantage of using a PCB is scalability and mass production. Once a circuit design is finalized and tested on a breadboard, transitioning to a PCB layout allows for easy replication and mass production. PCB manufacturing processes enable consistent and efficient production of multiple identical circuits, making it ideal for commercial or large-scale applications.

Lastly, PCBs offer a level of professionalism and aesthetics to the final product. The neatly arranged components and the absence of visible wires contribute to a clean and polished appearance. This is particularly relevant when designing devices for presentations, demonstrations, or commercial purposes, where visual appeal can significantly impact the perception of the product's quality and reliability.

In conclusion, using a PCB to mount and connect components offers numerous advantages over relying solely on a breadboard and wires. The enhanced reliability, compact layout, improved electrical performance, scalability, and professional aesthetics make PCBs the preferred choice for creating robust and efficient electronic circuits.
\begin{figure}[h]
\FloatBarrier
         \centering
        \includegraphics[width=1\textwidth]{pcb}
   
        \caption{Schematic Capture of the Interface card}
        \label{Schematic Capture of the Interface card}
\FloatBarrier
    \end{figure}
\FloatBarrier
Using Proteus, I began by designing the circuit in the schematic stage. The software provided an intuitive interface that allowed me to depict the components, their interconnections, and the overall circuit architecture. This schematic representation served as a blueprint for the circuit's functionality and ensured accurate translation into the physical layout.
\begin{figure}[h]
\FloatBarrier
         \centering
        \includegraphics[width=1\textwidth]{pcb1}
   
        \caption{PCB layout of the Interface card}
        \label{PCB layout of the Interface card}
\FloatBarrier
    \end{figure}
\FloatBarrier

With the schematic design in place, I seamlessly transitioned to the PCB layout stage using Proteus. The software offered a user-friendly environment where I could place components, define their physical arrangement on the PCB, and route traces to establish the required electrical connections. Proteus also allowed me to define design rules and constraints, ensuring proper spacing, clearance, and signal integrity.
In conclusion, by leveraging Proteus, I successfully designed and transformed the circuit from a schematic to a PCB layout. The software's capabilities, including schematic design, component placement, and trace routing, enabled me to create a well-structured and functional PCB design. Proteus enhanced the overall efficiency, accuracy, and reliability of the design process, resulting in a PCB ready for fabrication.
%\end{document}

