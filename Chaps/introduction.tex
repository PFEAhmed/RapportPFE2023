%\documentclass{report}
%\usepackage[left=1.25in,right=1.25in,top=1in,bottom=1.5in]{geometry} % set different margins
\usepackage{placeins}
\usepackage{graphicx}
\usepackage{hyperref}
\usepackage{bookmark}
\usepackage{titlesec}
\usepackage[T1]{fontenc}
%\usepackage{amsmath}
\usepackage{tabularx}
\usepackage{ifpdf}
\usepackage{ae}
\usepackage{tikz}
\usepackage{colortbl}
\usepackage{caption}

\usepackage{wrapfig}
%\usepackage{../Packages/tikz-uml}% this package is for uml diagrams
\usepackage{Packages/tikz-uml}% this package is for uml diagrams
\usepackage{caption} % load caption package
\usepackage{enumitem}
\usepackage{listings}
\usepackage{amsmath}
\usepackage{subcaption}
\usepackage{xcolor}
\usepackage{listings}
\usepackage{tcolorbox}

\tcbuselibrary{listings}
\definecolor{LightGray}{gray}{0.95}
\usepackage{tcolorbox}
\definecolor{dkgreen}{RGB}{0,128,0}
\NewTotalTCBox{\commandbox}{ s v }
{verbatim,colupper=white,colback=black!75!white,colframe=black}
{\IfBooleanT{#1}{\textcolor{red}{\ttfamily\bfseries > }}%
\lstinline[language=command.com,keywordstyle=\color{blue!35!white}\bfseries]^#2^}
\lstset{ 
  backgroundcolor=\color{LightGray},
  basicstyle=\ttfamily,
  breaklines=true,
  captionpos=b,
  language=bash,
  frame=single
}


\tcbset{colframe=blue!50!black,colback=white,colupper=black,
fonttitle=\bfseries,nobeforeafter,center title}
\definecolor{mauve}{RGB}{224,176,255}

\usetikzlibrary{positioning}
\lstset{frame=tb,
  language=Python,
  aboveskip=3mm,
  belowskip=3mm,
  showstringspaces=false,
  columns=flexible,
  basicstyle={\small\ttfamily},
  numbers=none,
  numberstyle=\tiny\color{gray},
  keywordstyle=\color{blue},
  commentstyle=\color{dkgreen},
  stringstyle=\color{mauve},
  breaklines=true,
  breakatwhitespace=true,
  tabsize=3
}
\setcounter{secnumdepth}{4}
\setcounter{tocdepth}{4}
\addcontentsline{toc}{section}{Introduction}

\renewcommand{\thesection}{\Roman{section}} 
\renewcommand{\thesubsection}{\arabic{subsection}}
\titlespacing{\subsection}{5pt}{1em}{0pt}
\graphicspath{ {./Ilustrations}{./}{./images}}
%\graphicspath{ {../Ilustrations}{../images}}

\tikzstyle{usecase}=[ellipse, fill=white, draw=black, thick, inner sep=2pt, text centered]
%\begin{document}
 \newpage
\setcounter{page}{1}
\begin{center}
% -----------------------------
% Introduction Page
% -----------------------------
\section*{Introduction}
\end{center}
The lack of precision in manufacturing has been a persistent challenge for businesses of all sizes, from the largest manufacturing plants to the most humble workshops.
Manufacturing businesses are often faced with the challenge of maintaining consistent precision in their production processes. The slightest deviation from the required standards can result in defective products, which can cause harm to the business's reputation and decrease sales. The causes of lack of precision can be numerous, ranging from technical glitches in production equipment to human error in the assembly line. Furthermore, production inefficiencies can also lead to a lack of precision, as poorly designed or executed manufacturing processes can result in inconsistent products.

In addition to the negative consequences of producing defective products, the costs associated with reworking, scrapping, or returning products can add up quickly. These expenses can negatively impact the company's bottom line and can even jeopardize the company's financial stability in the long run. As a result, it is critical for manufacturing businesses to implement rigorous quality control measures to ensure that products meet or exceed the required standards consistently. [2]
One solution that has shown great promise is the use of artificial intelligence and machine learning algorithms, such as neural networks, to optimize production processes and improve product quality. 

Neural networks and other artificial intelligence technologies have the potential to revolutionize the manufacturing industry by enabling businesses to identify and address issues more quickly and efficiently. With the ability to analyze vast amounts of data in real-time, these technologies can help businesses make informed decisions about how to improve their processes and reduce the risk of defective products. Moreover, by reducing the costs associated with reworking, scrapping, or returning products, these technologies can also help businesses improve their bottom line and increase profitability. 


At TOP TETHER, the company where I completed my internship, we were able to leverage the power of YOLOV5  to mitigate the problem of defective products and returns. By analyzing vast amounts of data from our manufacturing processes, we were able to identify patterns and correlations that were not immediately visible to the human eye. This enabled us to make products that were consistently of higher quality and closer to the required standards.


In conclusion, while the lack of precision in manufacturing remains a persistent challenge for many businesses, advances in technology offer promising solutions. By leveraging the power of artificial intelligence and machine learning algorithms, such as neural networks, businesses can gain deeper insights into their production processes, optimize their operations, and improve the quality of their products. As a result, they can reduce the risk of defective products and associated costs, while also improving their reputation and bottom line.

\newpage

%\end{document}